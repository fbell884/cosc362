%\documentclass[12pt,cancelspace]{exam}
\documentclass[12pt,answers]{exam}
\usepackage{color,geometry}
\usepackage{verbatim}
\usepackage{amsmath,amssymb}
\usepackage{graphicx}
\geometry{hmargin={1in,1in},vmargin={1in,1in}}
\begin{document}

\section*{Multiple Choice:}
\begin{questions}
%%%%% Question 1:
\question What does the "history" command do in Linux?
\begin{description}
\item[A.] Possible Answer \verb' It shows all of the previous users who have logged into a Linux server'
\item[B.] Possible Answer \verb' It shows all of the previous commands by a user.'
\item[C.] Possible Answer \verb' It shows all of the user's internet browser history'   
\item[D.] Possible Answer \verb' It gives a synopsis of the history of the united states'     
%answer: B     
\end{description}
\begin{solution}
(Answer B): The history commands gives a list of the previous commands by the user. The user can clear this.  
\end{solution}

%%%%% Question 2:
\question Which of the following is not a proper Git command?
\begin{description}
\item[A.] Possible Answer \verb' git status'
\item[B.] Possible Answer \verb' git push origin master'
\item[C.] Possible Answer \verb' git commit "This is my commit message"'    
\item[D.] Possible Answer \verb' git add .'     
%answer: C  
\end{description}
\begin{solution}
(Answer C): When committing files using Git, you need to use "-m" for the message. I.e., git commit -m "This is my commit message" .  
\end{solution}

%%%%% Question 3:
\question Which two commands are used to transfer files between two machines using sftp?
\begin{description}
\item[A.] Possible Answer \verb' get & post'
\item[B.] Possible Answer \verb' move & pull'
\item[C.] Possible Answer \verb' move & post'   
\item[D.] Possible Answer \verb' get & put'     
%answer: D     
\end{description}
\begin{solution}
(Answer D): The "get" command copies a file from the remote host to the local machine. The "put" command copies a file from the local machine and puts it on the remote file system.  
\end{solution}

%%%%% Question 4:
\question What is the proper way to begin bash script?
\begin{description}
\item[A.] Possible Answer \verb' pound bang slash bin slash bash'
\item[B.] Possible Answer \verb' bang bin slash bang slash bash'
\item[C.] Possible Answer \verb' bash pound slash bin slash bash'   
\item[D.] Possible Answer \verb' bin slash bang slash bash'     
%answer: A     
\end{description}
\begin{solution}
(Answer A): "pound bang slash bin slash bash" is the spelling out of "#!/bin/bash" which is how you should begin every bash script.  
\end{solution}

\end{questions}
\section*{Short Answer}
Give an example of how to make a short LaTeX document. 
\begin{solution}

"\documentclass[12pt]{article}
\author{Francis Bell}
\title{LaTeX Example}
\begin{document}
\maketitle

This is how you make a .tex document.

\end{document}"

\end{solution}
\end{document}
